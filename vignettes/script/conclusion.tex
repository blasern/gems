%--------------------------------------------------------------------------------
% Conclusion
%--------------------------------------------------------------------------------
\section{Conclusion} \label{conc}
In this paper we have presented the \proglang{R} package \pkg{gems}, which allows simulation from a directed acyclic multistate model. The \proglang{R} package \pkg{gems} is a flexible tool for investigating and evaluating health interventions. We have given detailed examples of the use of each function of \pkg{gems}, and an example of its use to evaluate the effect of reduced bleeding complications in TAVI patients on mortality. 

Several packages estimate and simulate Markov models, but we are not aware of any other packages that allow simulation from a multistate model with arbitrary transition-specific hazard functions. This flexibility in hazard functions improves its fit to data and allows to more accurately estimate the effects of different interventions. This package's inclusion of history-dependent transitions is also a major improvement on many traditional model structures. 

The \pkg{gems} package has some limitations, including the fact that a DAG is required. Sometimes it is useful to have models in which patients can return to a previous state. If this feature is not frequently required, the problem can be resolved by repeating the state in a DAG. Otherwise a different model structure is needed. A further limitation is that higher flexibility requires more intensive computation, compared to traditional models. Longitudinal processes could be incorporated in a joint model, and evaluations of the artificial cohorts could be further automated in future expansions. 

The \pkg{gems} package has useful functions for simulating hypothetical cohorts of patients based on a multistate model with general transition-specific hazard functions, and is a flexible and user-friendly tool for planning and evaluating public-health interventions.